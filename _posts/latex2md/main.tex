\documentclass[]{article}




\usepackage[utf8]{inputenc}
\usepackage{hyperref}
\usepackage{makecell}
\usepackage{url}
\usepackage{caption}
\usepackage{subcaption}
\usepackage{array}
\usepackage{times,color}
\usepackage[rflt]{floatflt}
\usepackage{epsfig,times,graphicx}
\usepackage{amsmath,amssymb,amsopn,algorithm,algorithmic,theorem,float,bbm,bm,enumerate,color,multirow}
\usepackage{rotating}
\usepackage{array}
%\usepackage{slashbox}
\usepackage{makecell}
\usepackage{multirow}
\usepackage{hhline}
\usepackage{xspace}
\usepackage{mathtools}
\usepackage{bm}

\newcommand\X{\mathbf{X}}
\newcommand\y{\mathbf{y}}
\newcommand\x{\mathbf{x}}
%\newcommand\bbeta{\boldsymbol{\beta}} %bold symbol is still not supported (added but not released in katex)
\newcommand\bbeta{{\beta}}
\newcommand{\norm}[2]{\|#1\|_{#2}}
\newcommand{\be}{\begin{eqnarray}}
\newcommand{\ee}{\end{eqnarray}}


%opening
\title{Structured High Dimensional Data Sharing Model}
\author{}



\begin{document}
	
		\maketitle 
		Consider the regression problem when we have a meaningful grouping in the sample space. 
		We want to exploit this knowledge and perform better in the prediction and parameter estimation tasks. 
		The (linear) data sharing (DS) model considers the following relation between covariates and the output:
		\be 
		y_i = \x_i (\bbeta_0 + \bbeta_g) + \epsilon_i 
		\ee 
		where $\bbeta_0$ and $\bbeta_g$ are shared (between all groups of samples) and the private (to group $g$) parameters respectively. 
		High dimensional structured data sharing model considers the DS when the number of features is much larger than samples and parameters have structure such as sparsity or group sparsity.
		We consider the general form of data sharing where the structure of both shared and private parameters can be characterized by any norm $R(\cdot)$. 		
\end{document}